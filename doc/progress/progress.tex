\documentclass{article}

\usepackage{fullpage}

\title{Reactive Gnu-Tools}

\author{Jake Eakle \and Joe Politz}

\begin{document}

\maketitle

\paragraph{Motivation}

Gnu tools like {\tt cp}, {\tt scp}, and {\tt find} are inherently
one-shot applications---they take a set of arguments and run to
completion accepting no further input from the user.  They are often
directed at large tasks, and time spent considering command line
arguments and consulting man pages can take away from the processing
time.  We propose several tweaks to common gnu filesystem tools to
make them more interactive, with specific focus on letting users help
the tool improve the efficiency of long tasks.

Consider some use cases:

\begin{itemize}
\item A common case of {\tt find} is that a user downloads a file
  \emph{somewhere} in a their home directory, and uses {\tt find}
  (from the command line or through a GUI), to search the entire
  directory for matching files.  Unless they are careful in their
  specification of directories to search, {\tt find} could end up in a
  costly search of obviously-incorrect directories, like the user's
  {\tt Music} folder, which contains thousands of mp3's, but little
  else of interest.  If the user had an interactive display of the
  places {\tt find} was searching, they could tell {\tt find} to skip
  certain directories.

\item A user is using {\tt cp} or {\tt scp} to copy a large number of
  files from one location to another, but can get back to their work
  sooner if they copy some files \emph{first}.  The ordering of files
  is opaque to the user and determined entirely by {\tt cp}.  Giving
  the user an interface to decide \emph{which} files should be
  transferred first, while the overall copy is ongoing, would be a
  useful extension.
\end{itemize}

\section{Progress --- Interface Designs}

\paragraph{Command Line}

We've focused our attention on {\tt find} so far, and come to a few
key points of interest.

\begin{itemize}
\item {\tt find} only provides feedback to users on files that are
  actually found.  Our application demands constant feedback from the
  application on the current path being searched.
\item In some uses of {\tt find}, the output on {\tt stdout} is piped
  to other applications, which conflicts with our goal of providing
  users with constant feedback.  We've considered and tried solutions
  involving interrupts and {\tt ncurses}, but don't have an entirely
  satisfactory solution yet.
\end{itemize}

\paragraph{Graphical}

[FILL]

\section{Progress --- Implementation}

We've spent time understanding {\tt find} and interposing our own code
into its pipeline of work.  Unsurprisingly, the bulk of {\tt find}'s
execution happens in a series of tight recursive calls, one of which
contains a tighter inner loop that iterates over the items in a
directory.  It is in this recursive process that we need to insert our
stops, accept user input, and unroll the recursion to the appropriate
level.

At the moment, we have the following workflow implemented:

\begin{enumerate}
\item {\tt find} receives {\tt SIGTSTP}, which calls a handler that
  sets a flag
\item The next time the {\tt process\_path} function (which sits at
  the top of the recursive call stack), is invoked, it checks this
  flag and drops the user to a prompt.
\item The user inputs the \emph{number of subdirectories} to jump up
  in the stack.
\item {\tt find} unrolls to the appropriate level without reporting
  any further results, and then continues on with the next directory
  entries in its implicit worklist.
\end{enumerate}

For example, here is a possible interaction:

\begin{verbatim}
> find /
/bin/bash
/bin/bunzip2
/bin/busybox
... (lots of output)
/var/lib/gems/1.8/gems/activerecord-2.3.3/lib/active_record/locale/en.yml
<program stops>
Enter number to skip: 5 <enter>
/var/lib/gems/1.8/gems/rack-1.0.0/...
<searching continues>
...
\end{verbatim}

\paragraph{Going Forward}

This is an interim solution (though it still is moderately useful),
until real-time feedback works well.  As noted above, we're looking
into using {\tt ncurses} in a separate thread to control input and
provide the user with feedback about what is currently being searched.

Our investigation into {\tt find} and playing with the source taught
us a lot about how gnu-tools work (and was part of the reason we were
interested in the project).  Now that we understand the architecture
of {\tt find} to some extent, we will have an easier time moving
forward with other tools.

\end{document}
